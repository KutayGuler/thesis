\documentclass{article}
\usepackage{graphicx} % Required for inserting images
\usepackage[
backend=biber,
style=numeric,
sorting=ynt
]{biblatex}
\usepackage{todonotes}

\addbibresource{bibliography.bib}

\title{Making A Roguelike Engine with SvelteKit}
\author{Kutay Güler}
\date{December 2023}

\begin{document}

\maketitle
\tableofcontents

\section{Introduction}
In every software project, whether it is a game, a game engine or a simple web application, there is a property which can cause a project to get derailed or follow a steady pace to its ultimate destination, and that property is called “scale.” 

Although getting derailed is a negative phrase, in the context of software engineering it could be a necessity, especially for game engines. Most popular game engines today have massive codebases with many people working on them, because their promise is the ability to create any type of game at any scale with only one tool. This is like inventing a train that can fly, travel underwater, go to the moon and of course, slide on iron rails. By its nature, it should derail from time to time. 

It sounds like there are only upsides to this invention since it can go anywhere but upon closer inspection, we can conclude that it is incorrect. The different environments this train will travel through will require many edge cases which will increase the literal size of the train and its maintenance cost. As for passengers, the experience will require some time to get used to and first timers will be definitely intimidated. 

This project’s aim is to create a straightforward train with its blueprints open to the public, that is, an open-source game engine specifically designed for roguelikes. The target audience is people who want to make roguelike games without having to learn coding but there will be grounds for programmers too. 

\subsection{Roguelike}
The term roguelike originated in a forum discussion to find an umbrella term for games that were similar to each other at the time. After three weeks of discussion, the “granddaddy of these games” Rogue, was appended with “-like” suffix to turn it into a game genre that we still use today \cite{roguelike-term}. 

Many roguelike games were released in the upcoming years, so it was time to define what a roguelike game is more concretely. In 2008, the International Roguelike Development Conference was held in Berlin, and gave birth to the Berlin Interpretation \cite{berlin}.

\section{Existing Solutions}
\subsection{Godot, Unity and Unreal}
These are the top three game engines for any genre. They are not specifically designed for roguelikes and have a longer learning process for the user since all of them require coding visually or manually \cite{godot}\cite{unity}\cite{unreal}.

\subsection{RPGMaker}
RPGMaker as the title suggests, is not an engine for roguelikes but our design decisions are similar. RPGMaker is an engine for RPGs (role playing games), and it has a low code environment, and it is the main inspiration for Roguelighter \cite{rpgmaker}.  

However, unlike Roguelighter, RPGMaker is proprietary software, and it has a UI layer for game data. Roguelighter has no UI for game data which eliminates any UI related issues that can arise like accessibility, readability, or consistency. 
 
Unity and Unreal is even used outside of game development industry. Their products are used in automobiles, architecture, engineering, and the film industry \cite{unity}\cite{unity-industry}\cite{unreal}.

\section{Technology}
\subsection{Svelte and SvelteKit}
Svelte is a JavaScript framework that leans on simplicity and performance with the power of its compiler-based architecture. An average Svelte project will have less lines of code and bundle size compared to its counterparts like React, Angular and Vue. Which makes it easier to maintain \cite{svelte-less}.

% TODO: change this description
SvelteKit is a full-stack application framework for Svelte. Although this project is not necessarily a full-stack application, a built-in router might be useful in the future, in case the software requires multiple views \cite{sveltekit}.

\subsection{WebGL}
WebGL (Web Graphics Library) is a JavaScript API for rendering high-performance interactive 3D and 2D graphics within any compatible web browser without the use of plug-ins. WebGL does so by introducing an API that closely conforms to OpenGL ES 2.0 that can be used in HTML canvas elements. This conformance makes it possible for the API to take advantage of hardware graphics acceleration provided by the user's device.
\todo{fix this}
% \cite{webgl} TODO: fix this

\subsection{ThreeJS}
ThreeJS is a canvas library which provides primitives and abstractions on top of WebGL to make building 3D projects faster, more readable, and more maintainable \cite{threejs}.

\subsection{Threlte}

\subsection{TypeScript}

\subsection{HOTScript}

\subsection{TailwindCSS}

\subsection{Monaco Editor}

\subsection{Tauri}

\section{Architecture}

\section{Website}
\subsection{Documentation}
\subsection{Interactive Tutorial}
\section{Conclusion}
\section{Future Work}
\printbibliography
\end{document}
